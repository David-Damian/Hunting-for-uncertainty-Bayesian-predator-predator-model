\usepackage{geometry}
\geometry{
 a4paper,
 total={195mm,280mm},
 left=15mm,
 right=15mm,
 top=15mm,
 bottom=15mm,
 }
% \renewcommand{\familydefault}{\sfdefault}
\usepackage{xcolor}
\usepackage[skins]{tcolorbox}

\usepackage[spanish, es-nodecimaldot]{babel}

\tcbset{mystyle/.style={
enhanced,
outer arc=0pt,
arc=0pt,
colframe=teal,
attach boxed title to top left,
boxed title style={
  colback=teal,
  outer arc=0pt,
  arc=0pt,
  },
title=Example~\thetcbcounter,
fonttitle=\sffamily
}
}
\newtcolorbox[auto counter]{Example}[1][]{
mystyle,
colback=white,
rightrule=0pt,
toprule=0pt,
title=EJERCICIO\text{},
}

\tcbset{mystyle1/.style={
enhanced,
outer arc=0pt,
arc=0pt,
colframe=red,
attach boxed title to top left,
boxed title style={
  colback=red,
  outer arc=0pt,
  arc=0pt,
  },
title=Nota~\thetcbcounter,
fonttitle=\sffamily
}
}
\newtcolorbox[auto counter]{Nota}[1][]{
mystyle1,
colback=white,
rightrule=0pt,
toprule=0pt,
title=Nota\text{},
}
\definecolor{coolblue}{RGB}{0,163,243}
\tcbset{mystyle2/.style={
enhanced,
outer arc=0pt,
arc=0pt,
colframe=coolblue,
colback=coolblue!20,
attach boxed title to top left,
boxed title style={
  colback=coolblue,
  outer arc=0pt,
  arc=0pt,
  },
title=supuestos~\thetcbcounter,
fonttitle=\sffamily
}
}
\newtcolorbox[auto counter]{supuestos}[1][]{
mystyle2,
colback=white,
rightrule=0pt,
toprule=0pt,
title=Supuestos del modelo LV\text{},
}

\definecolor{coolblue}{RGB}{0,163,243}
\tcbset{
  mystyle2/.style={
    enhanced,
    outer arc=0pt,
    arc=0pt,
    colframe=coolblue,
    colback=coolblue!20,
    attach boxed title to top left,
    boxed title style={
      colback=coolblue,
      outer arc=0pt,
      arc=0pt,
      },
    fonttitle=\sffamily,
    % Agregamos un \hfill para colocar el nombre del modelo al lado del título
    title={Modelo~\thetcbcounter\hfill#1},
  },
}

\newtcolorbox[auto counter]{modelo}[1][]{
  mystyle2={#1},
  colback=white,
  rightrule=0pt,
  toprule=0pt,
  % Eliminamos el título de aquí para evitar duplicación
}

\newcommand{\diff}[2]{\frac{\text{d}}{\text{d}#2}#1}
\newcommand{\normal}[2]{\textsf{\textbf{N}}(#1,#2)}
\newcommand{\lognormal}[2]{\textsf{\textbf{Log-normal}}(#1,#2)}

\newcommand{\norm}[1]{\left\lVert#1\right\rVert}
\renewcommand{\theenumi}{\alph{enumi})}
\usepackage{amsmath}%para alinear ecuaciones
\usepackage{amsfonts}%para letras tipo los reales
\usepackage{bbold}
\usepackage{graphicx}
\usepackage{caption}
\usepackage{subcaption}
\usepackage{amsmath}
\usepackage{enumitem} % enumera listas
\newtheorem{theorem}{Teorema}

\usepackage{booktabs}
\usepackage{longtable}
\usepackage{array}
\usepackage{multirow}
\usepackage{wrapfig}
\usepackage{float}
\usepackage{colortbl}
\usepackage{pdflscape}
\usepackage{tabu}
\usepackage{threeparttable}
\usepackage{threeparttablex}
\usepackage[normalem]{ulem}
\usepackage{makecell}
\usepackage{xcolor}

\newtheorem{definition}{Definición}
